% /usr/local/doc/tex-inputs/latex/notes/paper.tex

% An example showing how to prepare an article using AMSLaTeX.
% Stephen G. Simpson, Fall 1995.
% Please send any comments or questions to simpson@math.psu.edu.

% Lines beginning with a percent sign are comments.  LaTeX ignores them.

% Begin by declaring a document class and options.

\documentclass[12pt,oneside]{amsart}

% \documentclass{amsart} says to use the AMS article document class.
% [12pt,oneside] says to use the 12pt and oneside options.
% If you don't want these options, just say \documentstyle{amsart}.

% After the document class declaration comes the preamble.
% The preamble begins here.

   % First we activate any packages that we may need.
   %
   % The amssymb package provides \mathbb and other
   % math symbols.  The amsmath package provides sophisticated math
   % constructions.  The amsthm package provides \theoremstyle and
   % the \proof environment.
   %
   % The amsmath and amsthm packages are automatically activated by
   % \documentclass{amsart}, so there is no need to activate them here.

      \usepackage{amssymb}

   % Next we use \newtheorem to specify our theorem-like environments
   % (theorem, definition, etc.) and how to display and number them.
   %
   % Note: The \theoremstyle declarations affect the appearance of the
   % Theorems, Definitions, etc.

      \theoremstyle{plain}
      \newtheorem{theorem}{Theorem}[section]
      \newtheorem{lemma}[theorem]{Lemma}
      \newtheorem{corollary}[theorem]{Corollary}

      \theoremstyle{definition}
      \newtheorem{definition}[theorem]{Definition}

      \theoremstyle{remark}
      \newtheorem{remark}[theorem]{Remark}

   % The preamble is also a good place to define new commands and macros.
   % This part of the preamble is strictly optional according to your taste.

      \newcommand{\R}{{\mathbb R}}
      \newcommand{\nil}{\varnothing}

   % The following mysterious maneuver gets rid of AMS junk at the top
   % and bottom of the first page.

      \makeatletter
      \def\@setcopyright{}
      \def\serieslogo@{}
      \makeatother

% This ends the preamble.  We now proceed to the document itself.

\begin{document}

% First we specify the top matter (author, title, etc).
%
% Note: All of the top matter items are optional and can be omitted.
% But you will probably want to specify at least the author and title
% and perhaps an abstract.

   % author information

   % first author

   \author{Ting-Kam Leonard Wong}
   \address{University of Washington, Seattle, United States}
   \email{wongting@uw.edu}


   % title

   \title[Real Trees]{Coding continuous functions by real trees}

   % Note that the short title for running heads goes in square
   % brackets.  This is optional.  The long title goes in curly
   % braces.  In the long title, line breaks are indicated by \\.

   % abstract (optional)
%\begin{abstract}
%
%\end{abstract}

   % AMS subject classifications (used in AMS journals)
   % \subjclass{Primary 00A30; Secondary 00A22, 03E20}

   % AMS keywords (used in AMS journals)
   %\keywords{infinite, seminar}

   % acknowledge support, etc
   %\thanks{This research was partially supported by NSF grant
   %  DOA-123456789.}
   %\thanks{We would like to thank our colleagues for their helpful
   %  criticism.}

   % dedication
   %\dedicatory{Dedicated to Professor Donald Knuth on the occasion
   %  of his $100$th birthday}

   % today's date, or fill in whatever date you prefer
   \date{\today}

% This ends the top matter information.
% We can now tell LaTeX to display the top matter.

   \maketitle

% Having displayed the top matter, we now proceed to the body of the
% article.

% The body of the article is divided into sections.
% Each section begins with a \section command.

\section{Introduction}
In probability, the Galton-Watson process is a well-studied object. It models the
genealogical behaviors of a population. The population starts with a single individual.
In the next generation, the individual produces offspring according to a probability
distribution $\mu$ on $\{0, 1, 2, ...\}$. The offsprings then independently produce their
own offsprings according to the same distribution. The process continues until the population
becomes extinct (this happens with probability $1$ if ${\Bbb E} Z \leq 1$, where $Z$
is a random variable with distribution $\mu$). Naturally, this process can be
visualized by a (random) rooted tree. 

A less obvious fact is that a finite ordered tree $T$ gives rise to a contour function
$(C_t)_{t \geq 0}$. Assume that each edge of $T$ has length $1$. To draw this path, start
at the root $\rho$ and ``go around'' the tree, and $C_t$ is then the distance to $\rho$.
It is shown that if $T^{(n)}$ is a Galton-Watson tree whose offspring distribution satisfies
${\Bbb E}Z = 1$ and ${\mathrm{Var}}(Z) = 1$ and conditioned to have $n$ vertices,
and $C^{(n)}_t$ is the corresponding contour function, then
\begin{eqnarray} \label{Eqn 1}
\left(\frac{1}{\sqrt{2n}}C^{(n)}_{2t}\right)_{0 \leq t \leq 1} \overset{(d)}{\rightarrow} (e_t)_{0 \leq t \leq 1},
\end{eqnarray}
where $(e_t)_{t \geq 0}$ is a normalized Brownian excursion. This result suggests that
there is a strong relationship between random trees and Brownian excursions.
Using the concept of {\it real tree}, we can formulate the weak limit in (\ref{Eqn 1}) as a 
limit theorem for sequences of Galton-Watson trees. Details can be found in \cite{LG}.

It is interesting to see what these trees look like. In this project, we propose to write
codes that make construct a tree that approximates the real tree representing a given
continuous function.

\section{Mathematical background}
Let $g: [0, \infty) \rightarrow [0, \infty)$ be a continuous function with compact support
and satisfying $g(0) = 0$. For $s, t \geq 0$, define
\[
m_g(s, t) = \inf_{r \in [s \wedge t, s \vee t]} g(r),
\]
and
\[
d_g(s, t) = g(s) + g(t) - 2m_g(s, t).
\]
It can be shown that $d_g$ defines a pseudo-metric on $[0, \infty)$. Introduce the equivalence
relation $s \sim t$ if and only if $d_g(s, t) = 0$.

\begin{definition}
The real tree ${\mathcal T}_g$ of $g$ is define as the quotient space $[0, \infty) / \sim$. 
Let $p_g: [0, \infty) \rightarrow {\mathcal T}_g$ be the canonical projection. We view
$({\mathcal T}_g, d_g)$ as a rooted tree with root $\rho = p_g(0)$.
\end{definition}

We have the following approximation result.

\begin{theorem} \label{Thm 1}
Suppose $g_n$ is a sequence such that $g_n \rightarrow g$ uniformly. Then
\[
d_{GH}({\mathcal T}_{g_n}, {\mathcal T}_g) \rightarrow 0, \quad n \rightarrow \infty,
\]
where $d_{GH}$ denotes the Gromov-Hausdorff distance.
\end{theorem}

\section{Plan of project}
Based on Theorem \ref{Thm 1} we propose write an algorithm that does the following.

\begin{enumerate}
\item The input is a continuous function $g$ satisfying the properties above. Of course
we cannot store all the values of a continuous function. We assume, for example, that
$g(t)$ is defined for $t$ of the form $\frac{k}{2^n}$, where $n \geq 1$ is ``large''.
\item Fix an approximation level $n$. We replace $g$ by a function $g_n$ by linear
interpolation.
\item For the approximating function $g_n$, we construct the corresponding real tree ${\mathcal T}_{g_n}$.
For a piecewise linear function, ${\mathcal T}_{g_n}$ is a tree in the classical sense.
By Theorem \ref{Thm 1}, $T_{g_n}$ approximates ${\mathcal T}_g$ in the sense of 
Gromov-Hasudorff distance.
\item We plot the tree ${\mathcal T}_{g_n}$.
\end{enumerate}

\begin{thebibliography}{99}

\bibitem{LG}
     J.F. Le Gall, \emph{Random trees and applications},
     Probability Surveys 2 (2005), 245-311.
\end{thebibliography}

% Every LaTeX document must end with \end{document}.

\end{document}
